\section*{Aufgabe 11}
z.z.: 1) $Hom(V,W)$ abelsche Gruppe $(Hom(V,W),+)$\\
2) Skalarmultiplikation auf $Hom(V,W)$ definiert\\
zu 1)\\ z.z.: i) ist abgeschlossen auf +\\
ii) ist assoziativ\\
iii) hat Neutrales Element\\
iv) hat Inverses Element\\
v) ist kommutativ\\\\
zu i)\\
Sei $f,g \in Hom(V,W), x \in V\\
\rightarrow (f + g)(x) = f(x) + g(x)$ nach Definition der Addition auf $Hom(V,W)$\\
$\rightarrow f(x) + g(x) \in Hom(V,W)$ da Addition auf $\mathbb{R}^3$ abgeschlossen\\
$\rightarrow (f + g)(x) \in Hom(V,W)$\\
$\rightarrow$ $Hom(V,W)$ ist abgeschlossen auf +\\\\
zu ii)\\
Sei $f,g,h \in Hom(V,W), x \in V\\
\rightarrow (f + g + h)(x) = f(x) + g(x) + h(x)$ nach Definition der Addition auf $Hom(V,W)$\\
$\rightarrow (f(x) + g(x)) + h(x) \leftrightarrow f(x) + (g(x) + h(x))$ da Addition in $\mathbb{R}$ assoziativ\\
$\rightarrow ((f + g) + h)(x) \leftrightarrow (f + (g + h))(x)$ nach Definition der Addition auf $Hom(V,W)$\\
$\rightarrow$ ist assoziativ\\\\
zu iii)\\
Sei $f,g \in Hom(V,W), x \in V, g$ neutrales Element $ mit g(x) = 0$\\
$\rightarrow (f + g)(x) = f(x) + g(x)$ nach Definition Addition auf $Hom(V,W)$\\
$\rightarrow (f + g)(x) = f(x) + 0$\\
$\rightarrow (f + g)(x) = f(x)$\\
$\rightarrow$ Abbildung auf 0 ist neutrales Element\\
%$
%\begin{pmatrix}
%x \\
%y \\
%z
%\end{pmatrix}
%+
%\begin{pmatrix}
%0 \\
%0 \\
%0
%\end{pmatrix}
%=
%\begin{pmatrix}
%x + 0 \\
%y + 0 \\
%z + 0
%\end{pmatrix}
%=
%\begin{pmatrix}
%x \\
%y \\
%z
%\end{pmatrix}$\\
%$\rightarrow$ hat neutrales Element\\
\noindent zu iv)\\
Sei $f,f^{-1} \in Hom(V,W), x \in V, f^{-1} inverses Element mit f^{-1}(x) = -f(x)$\\
$\rightarrow (f + f^{-1})(x) = f(x) + f^{-1}(x)$ nach Definition Addition auf $Hom(V,W)$\\
$\rightarrow (f + f^{-1})(x) = f(x) + -f(x)$\\
$\rightarrow (f + f^{-1})(x) = 0$\\
$\rightarrow$ inverses Element $f^{-1}$ mit $f^{-1}(x) = -f(x)$\\\\
%inverses Element $u^{-1} = -u$\\
%$u =
%\begin{pmatrix}
%x \\
%y \\
%z
%\end{pmatrix}\\
%u + u^{-1} = u + (-u) = 
%\begin{pmatrix}
%x \\
%y \\
%z
%\end{pmatrix} + 
%\begin{pmatrix}
%-x \\
%-y \\
%-z
%\end{pmatrix}
%=
%\begin{pmatrix}
%x-x \\
%y-y \\
%z-z
%\end{pmatrix}
%=
%\begin{pmatrix}
%0 \\
%0 \\
%0
%\end{pmatrix}$\\
%$\rightarrow$ hat inverses Element\\
\noindent zu v)\\
kommutativ, da + in $\mathbb{R}$ kommutativ\\
Sei $f,g \in Hom(V,W), x \in V\\
\rightarrow (f + g)(x) = f(x) + g(x)$ nach Definition Addition auf $Hom(V,W)$\\
$\rightarrow (f + g)(x) = g(x) + f(x)$ da + auf $\mathbb{R}$ kommutativ\\
$\rightarrow (f + g)(x) = (g + f)(x)$ nach Definition Addition auf $Hom(V,W)$\\\\
zu 2)\\
trivial, da nach Aufgabenstellung Skalarmultiplikation f\"ur $Hom(V,W)$ definiert ist.\\\\
$\rightarrow$ Hom(V,W) ist $\mathbb{R}$-Vektorraum
